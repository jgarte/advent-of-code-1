\chapter{Bathroom Security}\label{bathroom-security}

\href{https://adventofcode.com/2016/day/2}{Link}

You arrive at \emph{Easter Bunny Headquarters} under cover of darkness.
However, you left in such a rush that you forgot to use the bathroom!
Fancy office buildings like this one usually have keypad locks on their
bathrooms, so you search the front desk for the code.

``In order to improve security,'' the document you find says, ``bathroom
codes will no longer be written down. Instead, please memorize and
follow the procedure below to access the bathrooms.''

The document goes on to explain that each button to be pressed can be
found by starting on the previous button and moving to adjacent buttons
on the keypad: \mintinline[]{idris}{U} moves up, \mintinline[]{idris}{D}
moves down, \mintinline[]{idris}{L} moves left, and
\mintinline[]{idris}{R} moves right. Each line of instructions
corresponds to one button, starting at the previous button (or, for the
first line, \emph{the ``5'' button}); press whatever button you're on at
the end of each line. If a move doesn't lead to a button, ignore it.

You can't hold it much longer, so you decide to figure out the code as
you walk to the bathroom. You picture a keypad like this:

\begin{minted}[]{text}
1 2 3
4 5 6
7 8 9
\end{minted}

Suppose your instructions are:

\begin{minted}[]{text}
ULL
RRDDD
LURDL
UUUUD
\end{minted}

\begin{itemize}
\tightlist
\item
  You start at ``5'' and move up (to ``2''), left (to ``1''), and left
  (you can't, and stay on ``1''), so the first button is
  \mintinline[]{idris}{1}.
\item
  Starting from the previous button (``1''), you move right twice (to
  ``3'') and then down three times (stopping at ``9'' after two moves
  and ignoring the third), ending up with \mintinline[]{idris}{9}.
\item
  Continuing from ``9'', you move left, up, right, down, and left,
  ending with \mintinline[]{idris}{8}.
\item
  Finally, you move up four times (stopping at ``2''), then down once,
  ending with \mintinline[]{idris}{5}.
\end{itemize}

So, in this example, the bathroom code is \mintinline[]{idris}{1985}.

Your puzzle input is the instructions from the document you found at the
front desk.

\section{Module Declaration and
Imports}\label{module-declaration-and-imports}

\begin{minted}[]{idris}
||| Day 2: Bathroom Security
module Data.Advent.Day02

import public Data.Advent.Day
import public Data.Ix

import Data.Vect

import public Lightyear
import public Lightyear.Char
import public Lightyear.Strings
\end{minted}

\section{Data Types}\label{data-types}

\begin{minted}[]{idris}
%access public export

||| Up, down, left or right.
data Instruction = ||| Up
                   U
                 | ||| Down
                   D
                 | ||| Left
                   L
                 | ||| Right
                   R

||| A single digit, i.e. a number strictly less than ten.
Digit : Type
Digit = Fin 10

implementation Show Digit where
  show = show . finToInteger

implementation [showDigits] Show (List Digit) where
  show = concatMap show

||| A pair of coordinates on the keypad, `(x, y)`.
Coordinates : Type
Coordinates = (Fin 3, Fin 3)
\end{minted}

\newpage

\section{Parsers}\label{parsers}

\begin{minted}[]{idris}
%access export

up : Parser Instruction
up = char 'U' *> pure U <?> "up"

down : Parser Instruction
down = char 'D' *> pure D <?> "down"

left : Parser Instruction
left = char 'L' *> pure L <?> "left"

right : Parser Instruction
right = char 'R' *> pure R <?> "right"

instruction : Parser Instruction
instruction = up <|> down <|> left <|> right <?> "up, down, left or right"

partial instructions : Parser (List Instruction)
instructions = some instruction <* (skip endOfLine <|> eof)
\end{minted}

\section{Part One}\label{part-one}

\begin{quote}
  What is the bathroom code?
\end{quote}

\begin{minted}[]{idris}
namespace PartOne

  ||| A keypad like this:
  |||
  ||| ```
  ||| 1 2 3
  ||| 4 5 6
  ||| 7 8 9
  ||| ```
  keypad : Vect 3 (Vect 3 Digit)
  keypad = [ [1, 2, 3],
             [4, 5, 6],
             [7, 8, 9] ]

  move : Coordinates -> Instruction -> Coordinates
  move (x, y) U = (x, pred y)
  move (x, y) D = (x, succ y)
  move (x, y) L = (pred x, y)
  move (x, y) R = (succ x, y)
\end{minted}

\newpage

\begin{minted}[]{idris}
  button : Coordinates -> List Instruction -> (Coordinates, Digit)
  button loc@(x, y) [] = (loc, index x (index y keypad))
  button loc (i :: is) = button (move loc i) is

partial partOne : List (List Instruction) -> String
partOne = show @{showDigits} . go ((1,1), [])
  where
    go : (Coordinates, List Digit) -> List (List Instruction) -> List Digit
    go (_, ds) []            = reverse ds
    go (loc, ds) (is :: iis) = let (loc', d) = PartOne.button loc is in
                                   go (loc', d :: ds) iis

namespace PartOne

  ||| ```idris example
  ||| example
  ||| ```
  partial example : String
  example = fromEither $ partOne <$>
            parse (some instructions) "ULL\nRRDDD\nLURDL\nUUUUD"
\end{minted}

\section{Part Two}\label{part-two}

You finally arrive at the bathroom (it's a several minute walk from the
lobby so visitors can behold the many fancy conference rooms and water
coolers on this floor) and go to punch in the code. Much to your
bladder's dismay, the keypad is not at all like you imagined it.
Instead, you are confronted with the result of hundreds of man-hours of
bathroom-keypad-design meetings:

\begin{minted}[]{text}
    1
  2 3 4
5 6 7 8 9
  A B C
    D
\end{minted}

You still start at ``5'' and stop when you're at an edge, but given the
same instructions as above, the outcome is very different:

\begin{itemize}
\tightlist
\item
  You start at ``5'' and don't move at all (up and left are both edges),
  ending at \mintinline[]{idris}{5}.
\item
  Continuing from ``5'', you move right twice and down three times
  (through ``6'', ``7'', ``B'', ``D'', ``D''), ending at
  \mintinline[]{idris}{D}.
\item
  Then, from ``D'', you move five more times (through ``D'', ``B'',
  ``C'', ``C'', ``B''), ending at \mintinline[]{idris}{B}.
\item
  Finally, after five more moves, you end at \mintinline[]{idris}{3}.
\end{itemize}

\newpage

So, given the actual keypad layout, the code would be
\mintinline[]{idris}{5DB3}.

\begin{quote}
  Using the same instructions in your puzzle input, what is the correct
  \textit{bathroom code}?
\end{quote}

\begin{minted}[]{idris}
namespace PartTwo

  keypad : Vect 5 (n ** Vect n Char)
  keypad = [ (1 **           ['1'])
           , (3 **      ['2', '3', '4'])
           , (5 ** ['5', '6', '7', '8', '9'])
           , (3 **      ['A', 'B', 'C'])
           , (1 **           ['D'])
           ]

  -- NOTE: This will wrap at the bounds, which might be unexpected.
  partial convert : (n : Nat) -> Fin m -> Fin n
  convert (S j) fm {m} =
      let delta = half $ if S j > m
                            then S j `minus` m
                            else m `minus` S j in
          the (Fin (S j)) $ fromNat $ finToNat fm `f` delta
    where
      f : Nat -> Nat -> Nat
      f = if S j > m then plus else minus
      partial half : Nat -> Nat
      half = flip div 2

  canMoveVertically : (Fin (S k), Fin 5) -> Instruction -> Bool
  canMoveVertically (x, y) i with ((finToNat x, finToNat y))
    canMoveVertically (x, y) U | (col, row) =
        case row of
             Z                   => False
             S Z                 => col == 1
             S (S Z)             => inRange (1,3) col
             _                   => True
    canMoveVertically (x, y) D | (col, row) =
        case row of
             S (S Z)             => inRange (1,3) col
             S (S (S Z))         => col == 1
             S (S (S (S Z)))     => False
             _                   => True
    canMoveVertically _ _ | _ = True
\end{minted}

\newpage

\begin{minted}[]{idris}
  partial move : (Fin (S k), Fin 5) -> Instruction ->
    ((n ** Fin n), Fin 5)
  move (x, y) U = if canMoveVertically (x, y) U
                     then let n = fst (index (pred y) keypad) in
                              ((n ** convert n x), pred y)
                     else ((_ ** x), y)
  move (x, y) D = if canMoveVertically (x, y) D
                     then let n = fst (index (succ y) keypad) in
                              ((n ** convert n x), succ y)
                     else ((_ ** x), y)
  move (x, y) L = let n = fst (index y keypad) in
                      ((n ** convert n (pred x)), y)
  move (x, y) R = let n = fst (index y keypad) in
                      ((n ** convert n (succ x)), y)

  partial button : (Fin (S k), Fin 5) -> List Instruction ->
    (((n ** Fin n), Fin 5), Char)
  button loc@(x, y) [] =
      let (n ** row) = index y PartTwo.keypad
          xx = convert n x in
          (((n ** xx), y), index xx row)
  button loc (i :: is) =
      let ((S _ ** x), y) = move loc i in
          button (x, y) is

partial partTwo : List (List Instruction) -> String
partTwo = go (((5 ** 0),2), [])
  where
    partial go : (((n ** Fin n), Fin 5), List Char) ->
      List (List Instruction) -> String
    go (_, cs) []            = pack $ reverse cs
    go (loc, cs) (is :: iis) =
        let ((S k ** xx), y) = loc
            (loc', c) = PartTwo.button (xx, y) {k=k} is in
            go (loc', c :: cs) iis

namespace PartTwo

  ||| ```idris example
  ||| PartTwo.example
  ||| ```
  partial example : String
  example = fromEither $ partTwo <$>
            parse (some instructions) "ULL\nRRDDD\nLURDL\nUUUUD"
\end{minted}

\section{Main}\label{main}

\begin{minted}[]{idris}
namespace Main

  partial main : IO ()
  main = runDay $ MkDay 2 (some instructions)
         (pure . partOne)
         (pure . partTwo)
\end{minted}
