\documentclass[b5paper,twoside]{amsbook}
\usepackage{lmodern}
\usepackage{amssymb,amsmath}
\usepackage{ifxetex,ifluatex}
\usepackage{fixltx2e} % provides \textsubscript
\ifnum 0\ifxetex 1\fi\ifluatex 1\fi=0 % if pdftex
  \usepackage[T1]{fontenc}
  \usepackage[utf8]{inputenc}
\else % if luatex or xelatex
  \ifxetex
    \usepackage{mathspec}
  \else
    \usepackage{fontspec}
  \fi
  \defaultfontfeatures{Ligatures=TeX,Scale=MatchLowercase}
\fi
% use upquote if available, for straight quotes in verbatim environments
\IfFileExists{upquote.sty}{\usepackage{upquote}}{}
% use microtype if available
\IfFileExists{microtype.sty}{%
\usepackage{microtype}
\UseMicrotypeSet[protrusion]{basicmath} % disable protrusion for tt fonts
}{}
\usepackage[margin=1in]{geometry}
\usepackage[unicode=true]{hyperref}
\hypersetup{
            pdftitle={Advent of Code 2016},
            pdfauthor={Eric Bailey},
            pdfborder={0 0 0},
            breaklinks=true}
\urlstyle{same}  % don't use monospace font for urls
\IfFileExists{parskip.sty}{%
\usepackage{parskip}
}{% else
\setlength{\parindent}{0pt}
\setlength{\parskip}{6pt plus 2pt minus 1pt}
}
\setlength{\emergencystretch}{3em}  % prevent overfull lines
\providecommand{\tightlist}{%
  \setlength{\itemsep}{0pt}\setlength{\parskip}{0pt}}
\setcounter{secnumdepth}{5}
% Redefines (sub)paragraphs to behave more like sections
\ifx\paragraph\undefined\else
\let\oldparagraph\paragraph
\renewcommand{\paragraph}[1]{\oldparagraph{#1}\mbox{}}
\fi
\ifx\subparagraph\undefined\else
\let\oldsubparagraph\subparagraph
\renewcommand{\subparagraph}[1]{\oldsubparagraph{#1}\mbox{}}
\fi

% set default figure placement to htbp
\makeatletter
\def\fps@figure{htbp}
\makeatother

\usepackage{minted}
\renewcommand{\chaptername}{Day}

\title{Advent of Code 2016}
\author{Eric Bailey}
\date{Winter 2016}

\begin{document}
\maketitle

\frontmatter

{
\setcounter{tocdepth}{2}
\tableofcontents
}
\mainmatter

\chapter{No Time for a Taxicab}\label{no-time-for-a-taxicab}

\href{https://adventofcode.com/2016/day/1}{Link}

Santa's sleigh uses a very high-precision clock to guide its movements,
and the clock's oscillator is regulated by stars. Unfortunately, the
stars have been stolen\ldots{} by the Easter Bunny. To save Christmas,
Santa needs you to retrieve all \textbf{fifty stars} by December 25th.

Collect stars by solving puzzles. Two puzzles will be made available on
each day in the advent calendar; the second puzzle is unlocked when you
complete the first. Each puzzle grants \textbf{one star}. Good luck!

You're airdropped near \emph{Easter Bunny Headquarters} in a city
somewhere. ``Near'', unfortunately, is as close as you can get - the
instructions on the Easter Bunny Recruiting Document the Elves
intercepted start here, and nobody had time to work them out further.

The Document indicates that you should start at the given coordinates
(where you just landed) and face North. Then, follow the provided
sequence: either turn left (\mintinline[]{idris}{L}) or right
(\mintinline[]{idris}{R}) 90 degrees, then walk forward the given number
of blocks, ending at a new intersection.

There's no time to follow such ridiculous instructions on foot, though,
so you take a moment and work out the destination. Given that you can
only walk on the
\href{https://en.wikipedia.org/wiki/Taxicab_geometry}{street grid of the
city}, how far is the shortest path to the destination?

For example:

\begin{itemize}
\tightlist
\item
  Following \mintinline[]{idris}{R2, L3} leaves you
  \mintinline[]{idris}{2} blocks East and \mintinline[]{idris}{3} blocks
  North, or \mintinline[]{idris}{5} blocks away.
\item
  \mintinline[]{idris}{R2, R2, R2} leaves you \mintinline[]{idris}{2}
  blocks due South of your starting position, which is
  \mintinline[]{idris}{2} blocks away.
\item
  \mintinline[]{idris}{R5, L5, R5, R3} leaves you
  \mintinline[]{idris}{12} blocks away.
\end{itemize}

\newpage

\section{Module Declaration and
Imports}\label{module-declaration-and-imports}

\begin{minted}[]{idris}
||| Day 1: No Time for a Taxicab
module Data.Advent.Day01
\end{minted}

For no compelling reason, I like to use
\href{https://www.haskell.org/arrows/}{arrows}, so import the requisite
modules.

\begin{minted}[]{idris}
import Control.Arrow
import Control.Category
import Data.Morphisms
import Data.SortedSet
\end{minted}

For parsing, use \href{https://github.com/ziman/lightyear}{Lightyear}.

\begin{minted}[]{idris}
import public Lightyear
import public Lightyear.Char
import public Lightyear.Strings
\end{minted}

\section{Data Types}\label{data-types}

Ensure both the type and data constructors are
\href{http://docs.idris-lang.org/en/latest/tutorial/modules.html\#meaning-for-data-types}{exported}
for all the following data types.

\begin{minted}[]{idris}
%access public export
\end{minted}

We'll need to keep track of which cardinal direction we're facing, so
define a data type \mintinline[]{idris}{Heading} to model that. A
heading is \mintinline[]{idris}{N}orth, \mintinline[]{idris}{E}ast,
\mintinline[]{idris}{S}outh or \mintinline[]{idris}{W}est.

\begin{minted}[]{idris}
||| A cardinal heading.
data Heading = ||| North
               N
             | ||| East
               E
             | ||| South
               S
             | ||| West
               W
\end{minted}

As per the \href{no-time-for-a-taxicab}{puzzle description}, the
instructions will tell us to go \mintinline[]{idris}{L}eft or
\mintinline[]{idris}{R}ight, so model \mintinline[]{idris}{Direction}s
accordingly.

\begin{minted}[]{idris}
||| A direction to walk in, left or right.
data Direction = ||| Left
                 L
               | ||| Right
                 R
\end{minted}

For convenience in the REPL, implement the \mintinline[]{idris}{Show}
\href{http://docs.idris-lang.org/en/latest/tutorial/interfaces.html}{interface}
for \mintinline[]{idris}{Direction}.

\begin{minted}[]{idris}
implementation Show Direction where
    show L = "Left"
    show R = "Right"
\end{minted}

\newpage

An \mintinline[]{idris}{Instruction}, e.g. \mintinline[]{idris}{R2}, is
comprised of a \mintinline[]{idris}{Direction} and a number of blocks to
walk in that direction.

As such, define an \mintinline[]{idris}{Instruction} as a pair of a
\mintinline[]{idris}{Direction} and an \mintinline[]{idris}{Integer}.

\begin{minted}[]{idris}
||| A direction and a number of blocks.
Instruction : Type
Instruction = (Direction, Integer)
\end{minted}

A \mintinline[]{idris}{Position} is a pair of a
\mintinline[]{idris}{Heading} and \mintinline[]{idris}{Coordinates},
such that the \mintinline[]{idris}{Heading} represents which cardinal
direction we're facing and the \mintinline[]{idris}{Coordinates}
describe where we're at in the
\href{https://en.wikipedia.org/wiki/Taxicab_geometry}{street grid}.

\begin{minted}[]{idris}
||| A pair of coordinates on the street grid, `(x, y)`.
Coordinates : Type
Coordinates = (Integer, Integer)

||| A heading and coordinates.
Position : Type
Position = (Heading, Coordinates)
\end{minted}

\section{Parsers}\label{parsers}

Ensure the type constructors of the following parsers are
\href{http://docs.idris-lang.org/en/latest/tutorial/modules.html\#meaning-for-data-types}{exported}.

\begin{minted}[]{idris}
%access export
\end{minted}

A \mintinline[]{idris}{Direction} is represented in the input by a
single character, \mintinline[]{idris}{'L'} or
\mintinline[]{idris}{'R'}.

\begin{minted}[]{idris}
direction : Parser Direction
direction = (char 'L' *> pure L) <|>|
            (char 'R' *> pure R) <?>
            "a direction"
\end{minted}

An \mintinline[]{idris}{Instruction} is represented as a
\mintinline[]{idris}{Direction} followed immediately by an integer.

\begin{minted}[]{idris}
||| Parse an instruction, i.e. a direction and
||| a number of blocks to walk in that direction.
partial instruction : Parser Instruction
instruction = [| MkPair direction integer |] <?> "an instruction"
\end{minted}

The instructions are given as a comma-separated list.

\begin{minted}[]{idris}
||| Parse a comma-separated list of `Instruction`s.
partial instructions : Parser (List Instruction)
instructions = commaSep instruction <* spaces <* eof <?>
               "a comma-separated instructions (at least one)"
\end{minted}

\newpage

\section{Logic}\label{logic}

\begin{minted}[]{idris}
||| Return the new heading after turning right.
||| @ initial the initial heading
turnRight : (initial : Heading) -> Heading
turnRight N = E
turnRight E = S
turnRight S = W
turnRight W = N

||| Return the new heading after turning left.
||| @ initial the initial heading
turnLeft : (initial : Heading) -> Heading
turnLeft N = W
turnLeft E = N
turnLeft S = E
turnLeft W = S

||| Return the new heading after turning
||| in the direction specifed by a given direction.
turn : Instruction -> (Position ~> Position)
turn (dir,_) = first . Mor $ case dir of { L => turnLeft; R => turnRight }

syntax "(~+" [n] ")" = plusN n

||| Add `n` to a given integer. Shorthand: `(~+ n)`
plusN : Integer -> (Integer ~> Integer)
plusN = Mor . (+)

syntax "(~-" [n] ")" = plusN n

||| Subtract `n` from a given integer. Shorthand: `(~- n)`
minusN : (n : Integer) -> (Integer ~> Integer)
minusN = Mor  . (flip (-))

||| Return a morphism `Position ~> Position` that follows a given instruction,
||| i.e. maps to the new position after moving the specified number of blocks
||| with the current heading from the current coordinates.
move : Instruction -> (Position ~> Position)
move (_,n) = Mor (second . go) &&& Mor id >>> Mor (uncurry applyMor)
  where
    go : Position -> (Coordinates ~> Coordinates)
    go (N,_) = second (~+ n)
    go (E,_) = first  (~+ n)
    go (S,_) = second (~- n)
    go (W,_) = first  (~- n)
\end{minted}

\newpage

\begin{minted}[]{idris}
||| Follow a given instruction and return the new position.
||| @ instruction the instruction to follow
||| @ pos the initial position
follow : (instruction : Instruction) -> (pos : Position) -> Position
follow ins = applyMor $ turn ins >>> move ins

private distance' : Coordinates -> Integer
distance' = abs . uncurry (+)

||| Return the distance from coordinates `(0,0)` after following
||| a given list of instructions, starting facing North.
||| @ instructions the list of instructions to follow
distance : (instructions : List Instruction) -> Integer
distance = distance' . snd . foldl (flip follow) (N, (0, 0))
\end{minted}

Define a generic main function for both parts of the puzzle that takes a
function \mintinline[]{idris}{f : List Instruction -> IO ()} and does as
follows:

\begin{itemize}
\tightlist
\item
  Read the input from \mintinline[]{idris}{input/day01.txt}
\item
  Parse a list of instructions from the input
\item
  Call \mintinline[]{idris}{f} on the list of instructions
\item
  If anything goes wrong, print the error and return
  \mintinline[]{idris}{()}
\end{itemize}

\begin{minted}[]{idris}
private partial main' : (f : List Instruction -> IO ()) -> IO ()
main' f = do Right str <- readFile "input/day01.txt"
               | Left err => printLn err
             case parse instructions str of
                  Right is => f is
                  Left err => printLn err
\end{minted}

\section{Part One}\label{part-one}

\begin{quote}
  \textit{How many blocks away} is Easter Bunny HQ?
\end{quote}

To compute the answer to Part One, \mintinline[]{idris}{262}, simply
call \mintinline[]{idris}{main' (printLn . distance)}, i.e.~compute the
\mintinline[]{idris}{distance} between the starting and final positions.

\begin{minted}[]{idris}
namespace PartOne
\end{minted}

\begin{minted}[]{idris}
    partial main : IO ()
    main = main' (printLn . distance)
\end{minted}

\newpage

\section{Part Two}\label{part-two}

Then, you notice the instructions continue on the back of the Recruiting
Document. Easter Bunny HQ is actually at the first location you visit
twice.

For example, if your instructions are
\mintinline[]{idris}{R8, R4, R4, R8}, the first location you visit twice
is \mintinline[]{idris}{4} blocks away, due East.

\begin{quote}
  How many blocks away is the \textit{first location you visit twice}?
\end{quote}

\begin{minted}[]{idris}
namespace PartTwo
\end{minted}

Computing the answer to Part Two, \mintinline[]{idris}{131}, is a bit
more complicated.

First, define a recursive function \mintinline[]{idris}{partTwo} to
process a list of instructions and return the distance between the
starting location and the first location visited twice.

To do that, maintain a sorted set of seen locations and for each block
walked per each instruction, and short circuit if any location has been
seen before.

Return \mintinline[]{idris}{Nothing} if given the empty instruction
list. Otherwise, return \mintinline[]{idris}{Just} the desired distance.

N.B. \mintinline[]{idris}{partTwo} short circuits at the instruction
level rather than block level, so it's possible a little extra work is
performed.

\begin{minted}[]{idris}
    partTwo : (instructions : List Instruction) ->
              (pos : Position) ->
              (seen : SortedSet Coordinates) ->
              Maybe Integer
    partTwo [] _ _                           = Nothing
    partTwo ((dir, len) :: is) (h, loc) seen =
        either (Just . distance')
               (partTwo is (follow (dir, len) (h, loc)))
               (foldl go (Right seen) [1..len])
      where
        go : Either Coordinates (SortedSet Coordinates) -> Integer ->
             Either Coordinates (SortedSet Coordinates)
        go dup@(Left _) _  = dup
        go (Right seen') n = let (_, loc') = follow (dir, n) (h, loc) in
                                 if contains loc' seen'
                                    then Left loc'
                                    else Right $ insert loc' seen'
\end{minted}

\begin{minted}[]{idris}
    partial main : IO ()
    main = main' $ \is => case partTwo is (N, (0, 0)) empty of
                               Nothing     => putStrLn "Failed!"
                               Just answer => printLn answer
\end{minted}

\newpage

\section{Main}\label{main}

Label and evaluate \mintinline[]{idris}{PartOne.main} and
\mintinline[]{idris}{PartTwo.main}.

\begin{minted}[]{idris}
namespace Main

    partial main : IO ()
    main = putStr "Part One: " *> PartOne.main *>
           putStr "Part Two: " *> PartTwo.main
\end{minted}

\(\qed\)

\end{document}
